\documentclass{beamer}
\usepackage{pgfpages}
\usepackage{hyperref}
\usepackage[ngerman]{babel}
\usepackage{graphicx}
\usepackage{url}
\usepackage[utf8x]{inputenc}
\usepackage{verbatim}

% no navigation icons
\setbeamertemplate{navigation symbols}{}
\usetheme{Copenhagen}
%gets rid of bottom navigation bars
\setbeamertemplate{footline}[page number]{}

%gets rid of navigation symbols
\setbeamertemplate{navigation symbols}{}
\usecolortheme{seahorse}
\usefonttheme{professionalfonts}

\title[Stackbasierte Programmiersprachen (WS2011)]{Eine Virtulle Maschine f\"ur Java Bytecode}
\author{Sebastian Rupl \and Josef Eisl \and Bernhard Urban}

\begin{document}
\frame{\titlepage}
%%%%%%
\begin{frame}
	\frametitle{Motivation}
	\begin{itemize}
		\item Java ist weit verbreitet
		\item \texttt{gforth} bereits auf vielen Plattformen vorhanden
		\item (mehr oder weniger) praktisch relevant
		\pause\item because we can (hopefully)
	\end{itemize}
\end{frame}
\begin{frame}
	\frametitle{Ziele}
	\begin{itemize}
		\item verschiedene Techniken f\"ur Threaded Code testen
		\item Unittests ($\Rightarrow$ Regressionstests, leichter zum
		Experimentieren)
		\item standard Classfiles verwenden
		\item Want:
		\begin{itemize}
			\item statische Methoden
			\item Objekterzeugung
			\item native Methoden (Forthwords?)
			\item minimale Java Api (e.g.\ \texttt{java.lang.System})
			\item Exception Handling
		\end{itemize}
		\item Not: Floating point, Inner Classes, Debugsupport, \dots
	\end{itemize}
\end{frame}
\begin{frame}
	\frametitle{Was bisher geschah\dots}
	\begin{center}
		demo
	\end{center}
\end{frame}
\end{document}
