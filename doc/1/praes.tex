\documentclass{beamer}
\usepackage{pgfpages}
\usepackage{hyperref}
\usepackage[english]{babel}
\usepackage{graphicx}
\usepackage{url}
\usepackage[utf8x]{inputenc}
\usepackage{verbatim}

% no navigation icons
\setbeamertemplate{navigation symbols}{}
\usetheme{Copenhagen}
%gets rid of bottom navigation bars
\setbeamertemplate{footline}[page number]{}

%gets rid of navigation symbols
\setbeamertemplate{navigation symbols}{}
\usecolortheme{seahorse}
\usefonttheme{professionalfonts}

\title{A Java Virtual Machine in Forth\\\small{Stack-Based Programming (WS2011)}}
\author{Sebastian Rumpl \and Josef Eisl \and Bernhard Urban}

\begin{document}
\frame{\titlepage}
%%%%%%
\begin{frame}
	\frametitle{Motivation}
	\begin{itemize}
		\item JavaVM is a stack-based virtual machine
		\item Java is wide spreaded in the real world
		\begin{itemize}
			\item but JavaVM is not restricted to Java (e.g.\ Scala)
		\end{itemize}
		\item JavaVM is well specified % (but well, ``well'')
		\item \texttt{gforth} already ported to many platforms
	\end{itemize}
\end{frame}
\begin{frame}
	\frametitle{JavaVM General}
	\begin{itemize}
		\item stack-based language, mostly typed instructions (e.g.\
		\texttt{int}, \texttt{float}, \texttt{references}, \dots)
		\item 32-bit based (i.e.\ \texttt{long} uses two entries on the stack)
		\item about 200 instructions $\Rightarrow$ one byte for opcode
		\item immediates encoded inside the program memory
		\item one data stack is used, similiar to Forth
		\item executables stored in so-called ``class files'' which includes:
		\begin{itemize}
			\item constant pool (e.g.\ strings, type information, \dots)
			\item code segment for each method
			\item \dots
		\end{itemize}
	\end{itemize}
\end{frame}
\begin{frame}
	\frametitle{Execution Cycle of a VM Instruction}
	\begin{itemize}
		\item \textbf{Fetch}: read one byte from instruction pointer
		\item \textbf{Decode}: load execution token from table
		\item \textbf{Execute}: execute it
		\item also called \texttt{NEXT} routine
		\item $\Rightarrow$ at the moment we use ``call threading''
	\end{itemize}
\end{frame}
\begin{frame}
	\frametitle{Goals}
	\begin{itemize}
		\item using standard classfiles (so we can use \texttt{javac})
		\item playing with various techniques for \emph{Threaded Code}
		\item using unittests, in order to make experimenting easier
		\pause
		\item want: 100\% compilance
		\pause
		\item want (serious):
		\begin{itemize}
			\item static methods
			\item object creation
			\item native methods (Forth words?)
			\item a minimal Java API (e.g.\ \texttt{java.lang.System})
			\item (simple) exception handling (i.e.\ without \texttt{finally})
		\end{itemize}
		\item not: floating point, inner classes, debugging support, UTF8, threads, \dots
	\end{itemize}
\end{frame}
\begin{frame}
	\frametitle{Challenges}
	\begin{itemize}
		\item dynamic loading of classes
		\item inheritance
		\item endianness
		\item UTF8 (we'll ignore it for now)
		\item Java API (GNU Classpath?), native calls
	\end{itemize}
\end{frame}
\begin{frame}
	\frametitle{Already implemented}
	\begin{itemize}
		\item classfile parsing
		\item core of execution (\texttt{NEXT})
		\item some instructions (like 10 or so\dots)
	\end{itemize}
	\begin{center}
		\pause demo
	\end{center}
\end{frame}
\end{document}
